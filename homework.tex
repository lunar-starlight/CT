% LTeX: enabled=false
% Kodiranje in podpora slovenščini
\usepackage[T1]{fontenc}        % kodiranje znakov v .pdf
\usepackage[english]{babel}

\usepackage{fontspec}
\usepackage{lualatex-math}
\usepackage{unicode-math}

% \setmainfont{TeX Gyre Pagella}
% \setmathfont{TeX Gyre Pagella Math}
\setmathfont{Latin Modern Math}
\setmathfont{Asana Math}[range={scr}]
\setmathfont{STIX Two Math-Regular}[range={bb}]
% \setmathfont{TeX Gyre Pagella Math}[range={8730}]
% \setmathfont{Asana Math}[range={"007B,"007D}]  % {}
\setmathfont{Asana Math}[range={8709, \setminus}]  % U+2205, emptyset

% Ta paket potrebujemo, ker amsart spreminja male črke v velike v naslovu,
% in tega ne zna pravilno delati s šumniki. Paket textcase ta problem odpravi.
% Glej: https://tex.stackexchange.com/questions/211305/problem-with-greek-title
\usepackage{textcase}

% Paketi za matematiko

\undef\eth
\undef\digamma
\undef\backepsilon
\usepackage{amsmath}  % razna okolja za poravnane enačbe ipd.
\usepackage{amsthm}   % definicije okolij za izreke, definicije, ...
% \usepackage{xypic}    % paket za diagrame

% \usepackage{mathtools}

\usepackage{fancyhdr}
\usepackage{extramarks}
\usepackage{enumerate}
\usepackage{tikz}
\usetikzlibrary{babel}
% \usepackage[plain]{algorithm}
% \usepackage{algpseudocode}
% \usepackage{listings}
\usepackage{minted}
\usepackage{hyperref}

\usetikzlibrary{cd,positioning,calc}

% quiver.sty
% A TikZ style for curved arrows of a fixed height, due to AndréC.
\tikzset{curve/.style={settings={#1},to path={(\tikztostart)
    .. controls ($(\tikztostart)!\pv{pos}!(\tikztotarget)!\pv{height}!270:(\tikztotarget)$)
    and ($(\tikztostart)!1-\pv{pos}!(\tikztotarget)!\pv{height}!270:(\tikztotarget)$)
    .. (\tikztotarget)\tikztonodes}},
    settings/.code={\tikzset{quiver/.cd,#1}
        \def\pv##1{\pgfkeysvalueof{/tikz/quiver/##1}}},
    quiver/.cd,pos/.initial=0.35,height/.initial=0}

% TikZ arrowhead/tail styles.
\tikzset{tail reversed/.code={\pgfsetarrowsstart{tikzcd to}}}
\tikzset{2tail/.code={\pgfsetarrowsstart{Implies[reversed]}}}
\tikzset{2tail reversed/.code={\pgfsetarrowsstart{Implies}}}
% TikZ arrow styles.
\tikzset{no body/.style={/tikz/dash pattern=on 0 off 1mm}}


%
% Basic Document Settings
%

\topmargin=-5mm
\evensidemargin=0cm
\oddsidemargin=0cm
\textwidth=17cm
\textheight=24cm
\headsep=6mm
\headheight=12.5pt

\linespread{1.1}

\pagestyle{fancy}
\lhead{\hmwkAuthorName}
\chead{\hmwkClass:\ \hmwkTitle}
\rhead{\firstxmark}
\lfoot{\lastxmark}
\cfoot{\thepage}

\renewcommand\headrulewidth{0.4pt}
\renewcommand\footrulewidth{0.4pt}

\setlength\parindent{0pt}

%
% Create Problem Sections
%

\newcommand{\enterProblemHeader}[1]{
    \nobreak\extramarks{}{Problem \arabic{#1} continued on next page\ldots}\nobreak{}
    \nobreak\extramarks{Problem \arabic{#1} (continued)}{Problem \arabic{#1} continued on next page\ldots}\nobreak{}
}

\newcommand{\exitProblemHeader}[1]{
    \nobreak\extramarks{Problem \arabic{#1} (continued)}{Problem \arabic{#1} continued on next page\ldots}\nobreak{}
    \stepcounter{#1}
    \nobreak\extramarks{Problem \arabic{#1}}{}\nobreak{}
}

\usepackage{xstring}
\setcounter{secnumdepth}{0}
\newcounter{partCounter}
\newcounter{solCounter}
\newcounter{homeworkProblemCounter}
\setcounter{homeworkProblemCounter}{1}
\nobreak\extramarks{Naloga \arabic{homeworkProblemCounter}}{}\nobreak{}

%
% Homework Problem Environment
%
% This environment takes an optional argument. When given, it will adjust the
% problem counter. This is useful for when the problems given for your
% assignment aren't sequential. See the last 3 problems of this template for an
% example.
%
\newenvironment{homeworkProblem}[1][-1]{
    \IfInteger{#1}
    {
        \ifnum#1>0
            \setcounter{homeworkProblemCounter}{#1}
        \fi
        \section{Problem \arabic{homeworkProblemCounter}}
    }{
        \section{Problem \arabic{homeworkProblemCounter}: #1}
    }
    \setcounter{partCounter}{1}
    \setcounter{solCounter}{1}
    \enterProblemHeader{homeworkProblemCounter}
}{
    \exitProblemHeader{homeworkProblemCounter}
}


%
% Title Page
%

\title{
    \vspace{2in}
    \textmd{\textbf{\hmwkClass:\ \hmwkTitle}}\\
    \normalsize\vspace{0.1in}\small{Deadline:\ \hmwkDueDate}\\
    % \vspace{0.1in}\large{\textit{\hmwkClassInstructor\ \hmwkClassTime}}
    \vspace{3in}
}

\author{\hmwkAuthorName}
\date{}

% \renewcommand{\part{}}[1]{\textbf{\large Part \Alph{partCounter}}\stepcounter{partCounter}\\}
\renewcommand{\part}{
    \vspace{10pt}
    \textbf{(\alph{partCounter})}
    \stepcounter{partCounter}
}

% Alias for the Solution section header
\newcommand{\solution}[1][*]{
    \vspace{10pt}
    \ifx#1*
        \textbf{Solution (\alph{solCounter})}
        \stepcounter{solCounter}
    \else
        \textbf{Solution}
    \fi
}

\newcommand{\example}[1][*]{
    \vspace{10pt}
    \ifx#1*
        \textbf{Example (\alph{solCounter})}
        \stepcounter{solCounter}
    \else
        \textbf{Example}
    \fi
}

\AtBeginDocument{
    \maketitle
    \pagebreak
}

\makeatletter
\def\mathcenterto#1#2{\mathclap{\phantom{#1}\mathclap{#2}}\phantom{#1}}
\let\old@widetilde\widetilde
\def\widetildeto#1#2{\mathcenterto{#2}{\old@widetilde{\mathcenterto{#1}{#2}}}}
\makeatother

\usepackage{scalerel}
\usepackage{stackengine,wasysym}

% \newcommand{\vphi}{\phi}
\renewcommand{\phi}{\varphi}
\newcommand{\eps}{\varepsilon}
\renewcommand{\hat}{\widehat}
\renewcommand{\tilde}{\widetilde}
\renewcommand{\bar}{\overline}
\newcommand{\subs}{\subseteq}
\newcommand{\nin}{\not\in}

\newcommand{\p}[1]{\left( {#1} \right)}
\renewcommand{\b}[1]{\left[ {#1} \right]}
\newcommand{\set}[2]{\left\{ #1 \mid #2 \right\}}
\newcommand{\newfrac}[2]{{}^{#1}\!/_{\!#2}}
\newcommand{\quot}[2]{\newfrac{#1}{#2}}
\DeclareMathOperator{\im}{im}
\DeclareMathOperator{\coker}{coker}
\DeclareMathOperator{\coim}{coim}
\DeclareMathOperator{\id}{1}
\newcommand{\op}{\mathrm{op}}
\newcommand{\mb}[1]{\mathbold{#1}}
\newcommand{\mf}[1]{\mathfrak{#1}}
\newcommand{\mc}[1]{\mathcal{#1}}

\newcommand{\for}[2]{\forall#1.\;#2}
\newcommand{\exist}[2]{\exists\;#1\smallni:\;#2}
\newcommand{\existi}[2]{\exists!\;#1\smallni:\;#2}

\renewcommand{\check}{ \(\checkmark\)}

\AtBeginDocument{
    \newcat{C}{\mc{C}}
    \newcat{cSet}{\mathbf{Set}}
}

\newcommand{\newcat}[2]{%
    \expandafter\DeclareRobustCommand\csname#1\endcsname{\csname@ifnextchar\endcsname\bgroup{\csname@@#1\endcsname}{\csname@#1\endcsname}}
    \expandafter\newcommand\csname@#1\endcsname{#2}
    \expandafter\newcommand\csname@@#1\endcsname[1]{#2{\p{##1}}}
}

\makeatletter
\newcommand{\oset}[3][0ex]{%
    \mathrel{\mathop{#3}\limits^{
            \vbox to#1{\kern-2\ex@
                \hbox{$\scriptstyle#2$}\vss}}}}
\makeatother

\newcommand{\plbk}[8]{
    \begin{tikzcd}[ampersand replacement=\&]
        #1 \& #2\\
        #3 \& #4
        \arrow[from=1-1, to=1-2, "#5"]
        \arrow[from=1-1, to=2-1, "#6"]
        \arrow[from=1-2, to=2-2, "#7"]
        \arrow[from=2-1, to=2-2, "#8"]
        \arrow[from=1-1, to=2-2, "\scalebox{2}{\(\lrcorner\)}"{anchor=center,pos=0},phantom]
    \end{tikzcd}
}
\newcommand{\psht}[8]{
    \begin{tikzcd}[ampersand replacement=\&]
        {#1} \& {#2}\\
        {#3} \& {#4}
        \arrow[from=1-1, to=1-2, "#5"]
        \arrow[from=1-1, to=2-1, "#6"]
        \arrow[from=1-2, to=2-2, "#7"]
        \arrow[from=2-1, to=2-2, "#8"]
        \arrow[from=2-2, to=1-1, "\scalebox{2}{\(\ulcorner\)}"{anchor=center,pos=0},phantom]
    \end{tikzcd}
}