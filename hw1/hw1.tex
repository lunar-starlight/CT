% LTeX: language=en-GB
\documentclass[a4paper]{article}

% LTeX: enabled=false
% Kodiranje in podpora slovenščini
\usepackage[T1]{fontenc}        % kodiranje znakov v .pdf
\usepackage[english]{babel}

\usepackage{fontspec}
\usepackage{lualatex-math}
\usepackage{unicode-math}

% \setmainfont{TeX Gyre Pagella}
% \setmathfont{TeX Gyre Pagella Math}
\setmathfont{Latin Modern Math}
\setmathfont{Asana Math}[range={scr}]
\setmathfont{STIX Two Math-Regular}[range={bb}]
% \setmathfont{TeX Gyre Pagella Math}[range={8730}]
% \setmathfont{Asana Math}[range={"007B,"007D}]  % {}
\setmathfont{Asana Math}[range={8709, \setminus}]  % U+2205, emptyset

% Ta paket potrebujemo, ker amsart spreminja male črke v velike v naslovu,
% in tega ne zna pravilno delati s šumniki. Paket textcase ta problem odpravi.
% Glej: https://tex.stackexchange.com/questions/211305/problem-with-greek-title
\usepackage{textcase}

% Paketi za matematiko

\undef\eth
\undef\digamma
\undef\backepsilon
\usepackage{amsmath}  % razna okolja za poravnane enačbe ipd.
\usepackage{amsthm}   % definicije okolij za izreke, definicije, ...
% \usepackage{xypic}    % paket za diagrame

% \usepackage{mathtools}

\usepackage{fancyhdr}
\usepackage{extramarks}
\usepackage{enumerate}
\usepackage{tikz}
\usetikzlibrary{babel}
% \usepackage[plain]{algorithm}
% \usepackage{algpseudocode}
% \usepackage{listings}
\usepackage{minted}
\usepackage{hyperref}

\usetikzlibrary{cd,positioning,calc}

% quiver.sty
% A TikZ style for curved arrows of a fixed height, due to AndréC.
\tikzset{curve/.style={settings={#1},to path={(\tikztostart)
    .. controls ($(\tikztostart)!\pv{pos}!(\tikztotarget)!\pv{height}!270:(\tikztotarget)$)
    and ($(\tikztostart)!1-\pv{pos}!(\tikztotarget)!\pv{height}!270:(\tikztotarget)$)
    .. (\tikztotarget)\tikztonodes}},
    settings/.code={\tikzset{quiver/.cd,#1}
        \def\pv##1{\pgfkeysvalueof{/tikz/quiver/##1}}},
    quiver/.cd,pos/.initial=0.35,height/.initial=0}

% TikZ arrowhead/tail styles.
\tikzset{tail reversed/.code={\pgfsetarrowsstart{tikzcd to}}}
\tikzset{2tail/.code={\pgfsetarrowsstart{Implies[reversed]}}}
\tikzset{2tail reversed/.code={\pgfsetarrowsstart{Implies}}}
% TikZ arrow styles.
\tikzset{no body/.style={/tikz/dash pattern=on 0 off 1mm}}


%
% Basic Document Settings
%

\topmargin=-5mm
\evensidemargin=0cm
\oddsidemargin=0cm
\textwidth=17cm
\textheight=24cm
\headsep=6mm
\headheight=12.5pt

\linespread{1.1}

\pagestyle{fancy}
\lhead{\hmwkAuthorName}
\chead{\hmwkClass:\ \hmwkTitle}
\rhead{\firstxmark}
\lfoot{\lastxmark}
\cfoot{\thepage}

\renewcommand\headrulewidth{0.4pt}
\renewcommand\footrulewidth{0.4pt}

% \setlength\parindent{0pt}

%
% Create Problem Sections
%

\newcommand{\enterProblemHeader}[1]{
    \nobreak\extramarks{}{Problem \arabic{#1} continued on next page\ldots}\nobreak{}
    \nobreak\extramarks{Problem \arabic{#1} (continued)}{Problem \arabic{#1} continued on next page\ldots}\nobreak{}
}

\newcommand{\exitProblemHeader}[1]{
    \nobreak\extramarks{Problem \arabic{#1} (continued)}{Problem \arabic{#1} continued on next page\ldots}\nobreak{}
    \stepcounter{#1}
    \nobreak\extramarks{Problem \arabic{#1}}{}\nobreak{}
}

\usepackage{xstring}
\setcounter{secnumdepth}{0}
\newcounter{partCounter}
\newcounter{solCounter}
\newcounter{homeworkProblemCounter}
\setcounter{homeworkProblemCounter}{1}
\nobreak\extramarks{Naloga \arabic{homeworkProblemCounter}}{}\nobreak{}

%
% Homework Problem Environment
%
% This environment takes an optional argument. When given, it will adjust the
% problem counter. This is useful for when the problems given for your
% assignment aren't sequential. See the last 3 problems of this template for an
% example.
%
\newenvironment{homeworkProblem}[1][-1]{
    \IfInteger{#1}
    {
        \ifnum#1>0
            \setcounter{homeworkProblemCounter}{#1}
        \fi
        \section{Problem \arabic{homeworkProblemCounter}}
    }{
        \section{Problem \arabic{homeworkProblemCounter}: #1}
    }
    \setcounter{partCounter}{1}
    \setcounter{solCounter}{1}
    \enterProblemHeader{homeworkProblemCounter}
}{
    \exitProblemHeader{homeworkProblemCounter}
}


%
% Title Page
%

\title{
    \vspace{2in}
    \textmd{\textbf{\hmwkClass:\ \hmwkTitle}}\\
    \normalsize\vspace{0.1in}\small{Deadline:\ \hmwkDueDate}\\
    % \vspace{0.1in}\large{\textit{\hmwkClassInstructor\ \hmwkClassTime}}
    \vspace{3in}
}

\author{\hmwkAuthorName}
\date{}

% \renewcommand{\part{}}[1]{\textbf{\large Part \Alph{partCounter}}\stepcounter{partCounter}\\}
\renewcommand{\part}{
    \vspace{10pt}
    \noindent
    \textbf{(\alph{partCounter})}
    \stepcounter{partCounter}
}

% Alias for the Solution section header
\newcommand{\solution}[1][*]{
    \vspace{10pt}
    \noindent
    \ifx#1*
        \textbf{Solution (\alph{solCounter})}
        \stepcounter{solCounter}
    \else
        \textbf{Solution}
    \fi
}

\newcommand{\example}[1][*]{
    \noindent
    \vspace{10pt}
    \ifx#1*
        \textbf{Example (\alph{solCounter})}
        \stepcounter{solCounter}
    \else
        \textbf{Example}
    \fi
}

\AtBeginDocument{
    \maketitle
    \pagebreak
}

\makeatletter
\def\mathcenterto#1#2{\mathclap{\phantom{#1}\mathclap{#2}}\phantom{#1}}
\let\old@widetilde\widetilde
\def\widetildeto#1#2{\mathcenterto{#2}{\old@widetilde{\mathcenterto{#1}{#2}}}}
\makeatother

\usepackage{scalerel}
\usepackage{stackengine,wasysym}

% \newcommand{\vphi}{\phi}
\renewcommand{\phi}{\varphi}
\newcommand{\eps}{\varepsilon}
\renewcommand{\hat}{\widehat}
\renewcommand{\tilde}{\widetilde}
\renewcommand{\bar}{\overline}
\newcommand{\subs}{\subseteq}
\newcommand{\nin}{\not\in}

\newcommand{\p}[1]{\left( {#1} \right)}
\renewcommand{\b}[1]{\left[ {#1} \right]}
\newcommand{\set}[2]{\left\{ #1 \mid #2 \right\}}
\newcommand{\newfrac}[2]{{}^{#1}\!/_{\!#2}}
\newcommand{\quot}[2]{\newfrac{#1}{#2}}
\DeclareMathOperator{\im}{im}
\DeclareMathOperator{\coker}{coker}
\DeclareMathOperator{\coim}{coim}
\DeclareMathOperator{\id}{1}
\newcommand{\op}{\mathrm{op}}
\newcommand{\mb}[1]{\mathbold{#1}}
\newcommand{\mf}[1]{\mathfrak{#1}}
\newcommand{\mc}[1]{\mathcal{#1}}

\newcommand{\for}[2]{\forall#1.\;#2}
\newcommand{\exist}[2]{\exists\;#1\smallni:\;#2}
\newcommand{\existi}[2]{\exists!\;#1\smallni:\;#2}

\renewcommand{\check}{ \(\checkmark\)}

\AtBeginDocument{
    \newcat{C}{\mc{C}}
    \newcat{cSet}{\mathbf{Set}}
}

\newcommand{\newcat}[2]{%
    \expandafter\DeclareRobustCommand\csname#1\endcsname{\csname@ifnextchar\endcsname\bgroup{\csname@@#1\endcsname}{\csname@#1\endcsname}}
    \expandafter\newcommand\csname@#1\endcsname{#2}
    \expandafter\newcommand\csname@@#1\endcsname[1]{#2{\p{##1}}}
}

\makeatletter
\newcommand{\oset}[3][0ex]{%
    \mathrel{\mathop{#3}\limits^{
            \vbox to#1{\kern-2\ex@
                \hbox{$\scriptstyle#2$}\vss}}}}
\makeatother

\newcommand{\plbk}[8]{
    \begin{tikzcd}[ampersand replacement=\&]
        #1 \& #2\\
        #3 \& #4
        \arrow[from=1-1, to=1-2, "#5"]
        \arrow[from=1-1, to=2-1, "#6"]
        \arrow[from=1-2, to=2-2, "#7"]
        \arrow[from=2-1, to=2-2, "#8"]
        \arrow[from=1-1, to=2-2, "\scalebox{2}{\(\lrcorner\)}"{anchor=center,pos=0},phantom]
    \end{tikzcd}
}
\newcommand{\psht}[8]{
    \begin{tikzcd}[ampersand replacement=\&]
        {#1} \& {#2}\\
        {#3} \& {#4}
        \arrow[from=1-1, to=1-2, "#5"]
        \arrow[from=1-1, to=2-1, "#6"]
        \arrow[from=1-2, to=2-2, "#7"]
        \arrow[from=2-1, to=2-2, "#8"]
        \arrow[from=2-2, to=1-1, "\scalebox{2}{\(\ulcorner\)}"{anchor=center,pos=0},phantom]
    \end{tikzcd}
}

\newcommand{\hmwkTitle}{First Homework}
\newcommand{\hmwkDueDate}{18.~11.~2022}
\newcommand{\hmwkClass}{Category Theory}
\newcommand{\hmwkClassTime}{}
\newcommand{\hmwkClassInstructor}{}
\newcommand{\hmwkAuthorName}{\textbf{Luna Strah},~27222025}

\def\tX{\tilde{X}}
\def\tY{\tilde{Y\,}}
\def\tZ{\tilde{Z}}
\def\tmf{t_{m,f}}

\begin{document}
\newcat{Cpar}{\mc{C}_{\mathrm{par}}}

\section{Preface:}
Throughout this homework I use covariant composition (so \((f;g)(x) = (fg)(x) = g(f(x))\)) on morphisms. Composition (and use) of functors and natural transformations will stay as usual.
% If desired I can change this.
I also denote the equivalence class of a partial map representative as \(\b{m, f}\).

I will also write \(ℓ : \p{m, f} ⊑ \p{m', f'}\) (and \(≡\)) when I talk about the (unique) map which makes the (in)equality hold.

\begin{homeworkProblem}
    Show that partial maps in \(\C\) form a category. That is, define a category \(\Cpar\) with \(|\Cpar| = |\C|\) and \(\Cpar{X, Y} = \) the collection of partial maps from \(X\) to \(Y\).

    \solution[]
    We define \(\Cpar\) via \(|\Cpar| ≔ |\C|\) and for \(X,Y ∈ |\Cpar|\) we define \[\Cpar{X, Y} ≔ \set{\b{f,m} ⊆ \C{X₀,Y}×\C{X₀,X}}{X₀ ∈ |\C|, m \text{ monic}}.\]
    It remains to define the composition and identity, and show that the identity and associativity laws hold for the above.

    Define \(\id_X ∈ \Cpar\) as \(\b{\id_X, \id_X}\) and the composite of partial maps \(\b{m, f}\) and \(\b{n, g}\) as follows:
    taking the pullback of the cospan \({X₀ \overset{f}{→} Y \overset{n}{←} Y₀}\) gives us the following diagram
    \[\begin{tikzcd}
        {\bar{X₀}} & {Y₀} & {Z}\\
        {X₀}       & {Y} \\
        {X}
        \arrow[from=1-1, to=1-2, "\bar{f}"]
        \arrow[from=1-1, to=2-1, "\bar{n}"]
        \arrow[from=2-1, to=2-2, "f"]
        \arrow[from=2-1, to=3-1, "m"]
        \arrow[from=1-2, to=1-3, "g"]
        \arrow[from=1-2, to=2-2, "n"]
        \arrow[from=1-1, to=2-2, "\scalebox{2}{\(\lrcorner\)}"{anchor=center,pos=0},phantom]
    \end{tikzcd}\]
    We define \(\b{m, f};\b{n, g} ≔ \b{\bar{n}m, \bar{f}g}\).
    The map \(\bar{n}m\) is monic, as \(\bar{n}\) is monic, since it's the pullback of a monic map.
    It remains to show that composition is well-defined according to \(≡\).

    Let \(\p{m₁,f₁}\) and \(\p{m₂,f₂}\) be two partial map representatives of the same map.
    Then the two compositions (via pullbacks) with \(\b{n,g}\) are \(\b{\bar{n}ᵢmᵢ, \bar{fᵢ}g}\).
    Because \(φ : \p{m₁,f₁} ≡ \p{m₂,f₂}\), we can form the following diagrams:
    \[
    \begin{tikzcd}
        {Pᵢ} & {Y₀} & {Z}\\
        {Xᵢ} & {Y} \\
        {X}
        \arrow[from=1-1, to=1-2, "\bar{fᵢ}"]
        \arrow[from=1-1, to=2-1, "\bar{n}ᵢ"]
        \arrow[from=2-1, to=2-2, "fᵢ"]
        \arrow[from=2-1, to=3-1, "mᵢ"]
        \arrow[from=1-2, to=1-3, "g"]
        \arrow[from=1-2, to=2-2, "m"]
        \arrow[from=1-1, to=2-2, "\scalebox{2}{\(\lrcorner\)}"{anchor=center,pos=0},phantom]
    \end{tikzcd}\qquad
    \begin{tikzcd}
        {P₁}\\
        {}   & {P₂} & {Y₀}\\
        {}   & {X₂} & {}
        \arrow[from=1-1, to=2-2, "ψ", dashed]
        \arrow[from=2-2, to=3-2, "\bar{n}₂"]
        \arrow[from=2-2, to=2-3, "\bar{f₂}"]
        \arrow[from=2-2, to=3-3, "\scalebox{2}{\(\lrcorner\)}"{anchor=center,pos=0},phantom]
        \arrow[from=1-1, to=2-3, "\bar{f₁}", curve={height=-20pt}]
        \arrow[from=1-1, to=3-2, "\bar{n}₁φ"', curve={height=14pt}]
    \end{tikzcd}\quad
    \begin{tikzcd}
        {P₂}\\
        {}   & {P₁} & {Y₀}\\
        {}   & {X₁} & {}
        \arrow[from=1-1, to=2-2, "ψ'", dashed]
        \arrow[from=2-2, to=3-2, "\bar{n}₁"]
        \arrow[from=2-2, to=2-3, "\bar{f₁}"]
        \arrow[from=2-2, to=3-3, "\scalebox{2}{\(\lrcorner\)}"{anchor=center,pos=0},phantom]
        \arrow[from=1-1, to=2-3, "\bar{f₂}", curve={height=-20pt}]
        \arrow[from=1-1, to=3-2, "\bar{n}₂φ^{-1}"', curve={height=14pt}]
    \end{tikzcd}
    \]
    The outer squares on the right two diagrams commute, thus we get the (unique) morphisms \(ψ\) and \(ψ'\) from the pullback square. This gives us the identities \(ψ\bar{n}₂ = \bar{n}₁φ\), \(ψ\bar{f₂} = \bar{f₁}\), \(ψ'\bar{n}₁ = \bar{n}₂φ^{-1}\), and \(ψ'\bar{f₁} = \bar{f₂}\), which are exactly the required identities to conclude both \(\p{\bar{n}₁m₁, \bar{f₁}g} ⊑ \p{\bar{n}₂m₂, \bar{f₂}g}\) and \(\p{\bar{n}₂m₂, \bar{f₂}g} ⊑ \p{\bar{n}₁m₁, \bar{f₁}g}\). We can similarly prove that composition is also well-defined on the other argument.

    Define the identity morphism on \(X ∈ |\Cpar|\) as \(\p{\id_X, \id_X}\). Pullback along an identity give us the same morphism, so \(\b{m,f};\b{\id_X, \id_X} = \b{\id_X;m, f;\id_X} = \b{m,f}\) and \(\b{\id_X, \id_X};\b{m,f} = \b{m;\id_X, \id_X;f} = \b{m,f}\).

    It remains to show associativity.
    Consider the pullback diagrams
    \[\begin{tikzcd}
        {P}      & {}   & {Z₀} & {W}\\
        {P_{XY}} & {Y₀} & {Z} \\
        {X₀}     & {Y} \\
        {X}
        \arrow[from=1-1, to=1-3, "p"]
        \arrow[from=1-1, to=2-1, "\bar{n}"]
        \arrow[from=2-1, to=2-2, "\bar{f}"]
        \arrow[from=2-1, to=3-1, "\bar{m}"]
        \arrow[from=3-1, to=3-2, "f"]
        \arrow[from=3-1, to=4-1, "i"]
        \arrow[from=2-2, to=2-3, "g"]
        \arrow[from=2-2, to=3-2, "m"]
        \arrow[from=1-3, to=1-4, "h"]
        \arrow[from=1-3, to=2-3, "n"]
        \arrow[from=1-1, to=2-2, "\scalebox{2}{\(\lrcorner\)}"{anchor=center,pos=0},phantom]
        \arrow[from=2-1, to=3-2, "\scalebox{2}{\(\lrcorner\)}"{anchor=center,pos=0},phantom]
    \end{tikzcd}\text{    and    }
    \begin{tikzcd}
        {P'}  & {P_{YZ}'} & {Z₀} & {W}\\
        {}   & {Y₀}     & {Z} \\
        {X₀} & {Y}     \\
        {X}
        \arrow[from=1-1, to=1-2, "\bar{f}"]
        \arrow[from=1-1, to=3-1, "q"]
        \arrow[from=1-2, to=1-3, "\bar{g}"]
        \arrow[from=1-2, to=2-2, "\bar{n}'"]
        \arrow[from=3-1, to=3-2, "f"]
        \arrow[from=3-1, to=4-1, "i"]
        \arrow[from=2-2, to=2-3, "g"]
        \arrow[from=2-2, to=3-2, "m"]
        \arrow[from=1-3, to=1-4, "h"]
        \arrow[from=1-3, to=2-3, "n"]
        \arrow[from=1-1, to=2-2, "\scalebox{2}{\(\lrcorner\)}"{anchor=center,pos=0},phantom]
        \arrow[from=1-2, to=2-3, "\scalebox{2}{\(\lrcorner\)}"{anchor=center,pos=0},phantom]
    \end{tikzcd}
    \]
    We wish to show that \(\p{\bar{n}\bar{m}i, ph} ≡ \p{qi, \bar{f}\bar{g}h}\).
    % Pullback laws from the following two diagrams give us one inequality, and by symmetry we can also derive the other one:
    Consider now the following sub-diagrams of the above:
    \[
        \begin{tikzcd}
            {P}\\
            {}   & {P_{YZ}'} & {Z₀}\\
            {}   & {Y₀}      & {Z}
            \arrow[from=1-1, to=2-2, "φ", dashed]
            \arrow[from=2-2, to=3-2, "\bar{n}'"]
            \arrow[from=2-2, to=2-3, "\bar{g}"]
            \arrow[from=2-3, to=3-3, "n"]
            \arrow[from=3-2, to=3-3, "g"]
            \arrow[from=2-2, to=3-3, "\scalebox{2}{\(\lrcorner\)}"{anchor=center,pos=0},phantom]
            \arrow[from=1-1, to=2-3, "p", curve={height=-20pt}]
            \arrow[from=1-1, to=3-2, "\bar{n}\bar{f}"', curve={height=14pt}]
        \end{tikzcd}\qquad
        \begin{tikzcd}
            {P}\\
            {}   & {P'} & {P_{YZ}'}\\
            {}   & {X₀} & {Y}
            \arrow[from=1-1, to=2-2, "ψ", dashed]
            \arrow[from=2-2, to=3-2, "q"]
            \arrow[from=2-2, to=2-3, "\bar{f}"]
            \arrow[from=2-3, to=3-3, "\bar{n}'m"]
            \arrow[from=3-2, to=3-3, "f"]
            \arrow[from=2-2, to=3-3, "\scalebox{2}{\(\lrcorner\)}"{anchor=center,pos=0},phantom]
            \arrow[from=1-1, to=2-3, "φ", curve={height=-20pt}]
            \arrow[from=1-1, to=3-2, "\bar{n}\bar{m}"', curve={height=14pt}]
        \end{tikzcd}
    \]
    The outer squares are present in the original diagrams, so they commute. Then by pullback laws the maps \(φ\) and \(ψ\) exist and \(ψ\) is a witness for \(\p{\bar{n}\bar{m}i, ph} ⊑ \p{qi, \bar{f}\bar{g}h}\) (since \(ψqi = \bar{n}\bar{m}i\) and \(ψ\bar{f}\bar{g}h = φ\bar{g}h = ph\)).
\end{homeworkProblem}
\newpage
\begin{homeworkProblem}
    Define an identity-on-objects faithful functor \(I : \C → \Cpar\).

    \solution[]
    Define the functor as follows:
    \begin{align*}
               I : \C &→ \Cpar\\
                    X &↦ X\\
        \p{f : X → Y} &↦ \b{\id_X, f} : I(X) → I(Y).
    \end{align*}

    We need to verify it respects identities and compositions. Let \(X,Y,Z ∈ |\C|\), \(f ∈ \C{X,Y}\), and \(g ∈ \C{Y,Z}\).
    \begin{align*}
        &I(\id_X) = \b{\id_X, \id_X} = \id_X ∈ \Cpar,\\
        &I(f;g) = \b{\id_X, f;g} = \b{\id_X,f};\b{\id_Y,g} = I(f);I(g).
    \end{align*}

    Thus, \(I\) is indeed a functor. Let now \(f\) and \(g\) both be from \(\C{X,Y}\), such that \(I(f) = I(g)\). Then we have \(φ : \p{\id_X,f} ≡ \p{\id_X, g}\). From the equation \(φ\id_X = \id_X\) it follows that \(φ = \id_X\). Then it follows from \(φf = g\) that \(f = g\) and \(I\) is faithful.

    We used the fact that a pullback along an identity is again the same morphism.
    Consider the following diagram, with the inner and outer squares commuting.
    \[\begin{tikzcd}
        {P}\\
        {}  & {X} & {X}\\
        {}  & {Y} & {Y}
        \arrow[from=1-1, to=2-2, "p", shift left=1]
        \arrow[from=1-1, to=2-2, "h"',shift right=1]
        \arrow[from=2-2, to=3-2, "f"]
        \arrow[from=2-2, to=2-3, "\id_X"]
        \arrow[from=2-3, to=3-3, "f"]
        \arrow[from=3-2, to=3-3, "\id_Y"]
        % \arrow[from=2-2, to=3-3, "\scalebox{2}{\(\lrcorner\)}"{anchor=center,pos=0},phantom]
        \arrow[from=1-1, to=2-3, "p", curve={height=-20pt}]
        \arrow[from=1-1, to=3-2, "q"', curve={height=14pt}]
    \end{tikzcd}\]
    Then the required map from \(P\) to \(X\) as required for a pullback exists (and is \(p\)).
    It is also unique, since if there is another map \(h\) such that the triangles commute, then \(p = h\) follows from the upper triangle.
\end{homeworkProblem}
\newpage
\begin{homeworkProblem}
    Suppose that every object \(X\) of \(\C\) has a partial map classifier. Show that the operation \(X ↦ \tX\) extends to a faithful functor \(\tilde{(-)} : \C → \C\) with respect to which the maps \((η_X)_{X ∈ |\C|}\) form the components of a natural transformation \(η : \id_{\C} ⇒ \tilde{(-)}\).

    \solution[]

    \textbf{Lemma:}
    The maps \(η_X\) are monic.
    \begin{proof}[Proof of lemma]
        Let \(f,g : X → Y\) be such that \(fη_Y = gη_Y\). Then consider the following diagram:
        \[\begin{tikzcd}
            {X}\\
            {} & {X} & {Y}\\
            {} & {X} & {\tY}
            \arrow[from=1-1, to=2-2, "φ", dashed]
            \arrow[from=2-2, to=2-3, "f"]
            \arrow[from=2-2, to=3-2, "\id_X"]
            \arrow[from=3-2, to=3-3, "t_{\id_X,f}"']
            \arrow[from=2-3, to=3-3, "η_Y"]
            \arrow[from=2-2, to=3-3, "\scalebox{2}{\(\lrcorner\)}"{anchor=center,pos=0},phantom]
            \arrow[from=1-1, to=2-3, "g", curve={height=-20pt}]
            \arrow[from=1-1, to=3-2, "\id_X"', curve={height=14pt}]
        \end{tikzcd}\]
        The inner square is a pullback, since it arises from the partial map classifier of \(\p{\id_X, f}\), and the outer square commutes by assumption (\(\id_X;t_{\id_X,f} = f;η_Y = g;η_Y\)).
        Then by pullback laws there exists a unique \(φ : X → X\) such that \(φ;\id_X = \id_X\) and \(φ;f = g\). From the first equation we get \(φ = \id_X\) which means that \(f = g\) follows and thus, \(η_Y\) is a mono.
    \end{proof}
    \begin{proof}
        Let \(f : X → Y\). Define then \(\tilde{f}\) to be the map \(t_{η_X, f}\) arising from the partial map classifier of the partial map representative \(\p{η_X, f}\) (as shown above, \(η_X\) is monic). This makes the diagram \(\begin{tikzcd}
            {X}         & {Y}\\
            {\tX} & {\tY}
            \arrow[from=1-1, to=1-2, "f"]
            \arrow[from=1-1, to=2-1, "η_X"]
            \arrow[from=2-1, to=2-2, "\tilde{f}"]
            \arrow[from=1-2, to=2-2, "η_Y"]
        \end{tikzcd}\) commute, which will satisfy the naturality condition later.

        Next we have to show \(\tilde{(-)}\) is a functor.
        We have already defined the functions \(F₀ : |\C| → |\C|\) and \(F_{1,X,Y} : \C{X,Y} → \C{\tX, \tY}\) appropriately, so we only need to verify that \(\tilde{\id_X} = \id_{\tX}\) and \(\tilde{f;g} = \tilde{f};\tilde{g}\).

        First consider the following diagrams:
        \[
            \begin{tikzcd}
                {X}   & {X}\\
                {\tX} & {\tX}
                \arrow[from=1-1, to=1-2, "\id_X"]
                \arrow[from=1-1, to=2-1, "η_X"]
                \arrow[from=2-1, to=2-2, "\tilde{\id_X}"]
                \arrow[from=1-2, to=2-2, "η_X"]
            \end{tikzcd}\qquad
            \begin{tikzcd}
                {X}   & {X}\\
                {\tX} & {\tX}
                \arrow[from=1-1, to=1-2, "\id_X"]
                \arrow[from=1-1, to=2-1, "η_X"]
                \arrow[from=2-1, to=2-2, "\id_{\tX}"]
                \arrow[from=1-2, to=2-2, "η_X"]
            \end{tikzcd}
        \]
        %TODO: elaborate?
        They are both pullbacks, so by uniqueness of \(t_{η_X, \id_X}\) they are equal, so \(\tilde{\id_X} = \id_{\tX}\).

        Next consider the following diagrams:
        \[
            \begin{tikzcd}
                {X}   & {Y}   & {Z}  \\
                {\tX} & {\tY} & {\tZ}
                \arrow[from=1-1, to=2-1, "η_X"]
                \arrow[from=1-1, to=1-2, "f"]
                \arrow[from=2-1, to=2-2, "\tilde{f}"]
                \arrow[from=1-2, to=2-2, "η_Y"]
                \arrow[from=1-2, to=1-3, "g"]
                \arrow[from=2-2, to=2-3, "\tilde{g}"]
                \arrow[from=1-3, to=2-3, "η_Z"]
            \end{tikzcd}\qquad
            \begin{tikzcd}
                {X}   & {Z}\\
                {\tX} & {\tZ}
                \arrow[from=1-1, to=1-2, "f;g"]
                \arrow[from=1-1, to=2-1, "η_X"]
                \arrow[from=2-1, to=2-2, "\tilde{f;g}"]
                \arrow[from=1-2, to=2-2, "η_Z"]
            \end{tikzcd}
        \]
        On the left diagram the inner two squares are pullbacks, thus by the pullback lemma so is the outer square. But then, as the right square is also a pullback diagram, we get \(\tilde{f;g} = \tilde{f};\tilde{g}\) by uniqueness via the partial map classifier.

        Together then \(\tilde{(-)}\) is a functor and \(η : \id_{\C} ⇒ \tilde{(-)}\) is a natural transformation.
    \end{proof}
\end{homeworkProblem}
\newpage
\begin{homeworkProblem}
    Suppose that \(Y\) has a partial map classifier. Show that, for any two representatives \(\p{m, f}\) and \(\p{m', f'}\) of partial maps from \(X\) to \(Y\), it holds that \(\p{m, f} ≡ \p{m', f'}\) if and only if \(\tmf = t_{m',f'}\).
    
    \solution[]
    \begin{proof}
        Consider the diagram
        \[\begin{tikzcd}
            {X₀} & {X₀'} & {Y}\\
            {X}  & {X}   & {\tY}
            \arrow[from=1-1, to=1-2, "i"]
            \arrow[from=1-1, to=2-1, "m'"]
            \arrow[from=2-1, to=2-2, "\id_X", Rightarrow, no head]
            \arrow[from=1-2, to=2-2, "m"]
            \arrow[from=1-1, to=1-3, "f"', curve={height=-28pt}]
            \arrow[from=1-2, to=1-3, "f'"]
            \arrow[from=1-3, to=2-3, "η_Y"]
            \arrow[from=2-2, to=2-3, "t'"]
            \arrow[from=1-2, to=2-3, "\scalebox{2}{\(\lrcorner\)}"{anchor=center,pos=0},phantom]
        \end{tikzcd}\]
        Then if \(i : \p{m, f} ≡ \p{m', f'}\) the above diagram commutes.
        We wish to show the left square is a pullback.
        Take \(X \overset{p}{←} Z \overset{q}{→} X₀'\), such that \(p = qm\).
        Then, since \(i\) is an iso \(qi^{-1}\) is a morphism from \(Z\) to \(X₀\).
        Because \(Y\) has a partial map classifier that morphism is also unique (as it maps into a pullback), so the left square is also a pullback.
        Then, by the pullback lemma the outer square is also a pullback.
        From there, we conclude that \(t = t'\), since \(t\) is the unique morphism, that makes that square a pullback.

        In the other direction, from the assumption that \(t = t'\) we see that the outer and right squares are pullbacks, and \(i : X₀ → X₀'\) exists because \(X₀'\) is a pullback and the diagram (without \(i\)) commutes. This \(i\) also makes the whole diagram commute.
        Then again by the pullback lemma we conclude that the left square is a pullback.
        Then \(i\) is an iso and \(i : \p{m, f} ≡ \p{m', f'}\).
    \end{proof}
\end{homeworkProblem}
\newpage
\begin{homeworkProblem}
    Show that an object \(Y\) has a partial map classifier if and only if the functor \(F ≔ \Cpar{-, Y} ◦ I^\op : \C^\op → \cSet\) is representable.
    (Here \(I^\op : \C^\op → \Cpar^\op\) is the opposite-category version of \(I : \C → \Cpar\)).

    \solution[]

    Let us first break down both definitions:
    \begin{itemize}
        \item \(\exist{\tY ∈ |\C|, η_Y : Y → \tY}{\for{\b{m,f} ∈ \Cpar{X, Y}}{\existi{\tmf ∈ \C{X, \tY}}{\text{the square is a pullback}}}}\).
        \item \(\exist{Z ∈ |\C|, α : \Cpar{-, Y}∘I^\op ⇒ \C{-, Z}}{α \text{ is a natural isomorphism}}\).
        \begin{itemize}
            \item For all \(X ∈ |\C|\) \(α_X\) is an iso
            \item For all \(f ∈ \C{X, X'}\) we have
                \[\begin{tikzcd}
                    {X}  && {\Cpar{X , Y}} & {\C{X , Z}}\\
                    {X'} && {\Cpar{X', Y}} & {\C{X', Z}}
                    \arrow[from=1-1, to=2-1, "f"]
                    \arrow[from=1-3, to=1-4, "α_X"]
                    \arrow[from=2-3, to=1-3, "Ff"]
                    \arrow[from=2-3, to=2-4, "α_{X'}"]
                    \arrow[from=2-4, to=1-4, "f^*"']
                \end{tikzcd}\]
                where \(Ff = \b{m, g} ↦ \b{\id_X, f};\b{m, g} = \b{\bar m, \bar f;g}\).
        \end{itemize}
    \end{itemize}

    The equivalence will obviously take \(\tY = Z\).
    \begin{proof}[\((⇒)\)]
        Define \(α_X\b{m, f} ≔ \tmf\), the unique map arising from the partial map classifier \(\tY\). It is well-defined by problem 4 (\(\b{m,f} = \b{m',f'} ⇒ \tmf = t_{m',f'}\)).

        Take \(\b{m,g} ∈ \Cpar{X',Y}\) and \(f ∈ \C{X,X'}\). Then we have \(\p{Ff;α_X}\b{m,g} = t_{\bar{m},\bar{f};g}\) and \(\p{α_{X'};f^*}\b{m,g} = f;t_{m,g}\).
        As before by pullback lemma on the diagram
        \[\begin{tikzcd}
            {X₀} & {X₀'} & {Y}  \\
            {X } & {X'}  & {\tY}
            \arrow[from=1-1, to=2-1, "\bar{m}"]
            \arrow[from=1-1, to=1-2, "\bar{f}"]
            \arrow[from=2-1, to=2-2, "f"]
            \arrow[from=1-2, to=2-2, "m"]
            \arrow[from=1-2, to=1-3, "g"]
            \arrow[from=2-2, to=2-3, "t_{m,g}"]
            \arrow[from=1-3, to=2-3, "η_Y"]
            \arrow[from=1-1, to=2-2, "\scalebox{2}{\(\lrcorner\)}"{anchor=center,pos=0},phantom]
            \arrow[from=1-2, to=2-3, "\scalebox{2}{\(\lrcorner\)}"{anchor=center,pos=0},phantom]
        \end{tikzcd}\]
        and uniqueness of \(t_{\bar{m}, \bar{f};g}\) we see that \(f;t_{m,g} = t_{\bar{m}, \bar{f};g}\).
        This makes the naturality square commute, so \(α\) is a natural transformation.

        Define \(β : \C{-, \tY} ⇒ F\) by \(β_Xh ≔ \b{p, q}\) where \(X \overset{p}{←} P \overset{q}{→} Y\) is the pullback of the cospan \(X \overset{h}{→} \tY \overset{η_Y}{←} Y\).

        Then \(\p{αβ}_Xh = t_{p, q} = h\) by uniqueness of \(t_{p,q}\) and \(\p{βα}_{X'}\b{m,g} = \b{m,g}\). Thus, \(α\) is a natural isomorphism.
    \end{proof}
    \begin{proof}[\((⇐)\)]
        Let now \(α : \Cpar{-, Y}∘I^\op ⇒ \C{-, Z}\) be a natural isomorphism and \(X \overset{m}{←} X₀ \overset{f}{→} Y\) be a partial map representative from \(X\) to \(Y\).

        Define \(η_Y ≔ α_Y\b{\id_Y,\id_Y}\) and \(\tmf ≔ α_X\b{m,f}\).

        Keep the following diagram in mind (note that we do not assert the commutativity or existence of any of the maps indicated by convention):
        \[\begin{tikzcd}
            {P}\\
            {} & {X₀} & {Y}\\
            {} & {X}  & {\tY}
            \arrow[from=1-1, to=2-2, "φ", dashed]
            \arrow[from=2-2, to=2-3, "f"]
            \arrow[from=2-2, to=3-2, "m"]
            \arrow[from=3-2, to=3-3, "\tmf"']
            \arrow[from=2-3, to=3-3, "η_Y"]
            \arrow[from=1-1, to=2-3, "p", curve={height=-20pt}]
            \arrow[from=1-1, to=3-2, "q"', curve={height=14pt}]
        \end{tikzcd}\]

        First we have to prove the inner square commutes. Consider the following diagrams that arise from the naturality of \(α\):
        \[\begin{tikzcd}
            {X₀} & {\Cpar{X₀, Y}} & {\C{X₀, Z}}\\
            {Y}  & {\Cpar{Y , Y}} & {\C{Y , Z}}
            \arrow[from=1-1, to=2-1, "f"]
            \arrow[from=1-2, to=1-3, "α_{X₀}"]
            \arrow[from=2-2, to=1-2, "Ff"]
            \arrow[from=2-2, to=2-3, "α_{Y}"]
            \arrow[from=2-3, to=1-3, "f^*"']
        \end{tikzcd}\qquad\qquad
        \begin{tikzcd}
            {X₀} & {\Cpar{X₀, Y}} & {\C{X₀, Z}}\\
            {X}  & {\Cpar{X , Y}} & {\C{X , Z}}
            \arrow[from=1-1, to=2-1, "m"]
            \arrow[from=1-2, to=1-3, "α_{X₀}"]
            \arrow[from=2-2, to=1-2, "Fm"]
            \arrow[from=2-2, to=2-3, "α_{X}"]
            \arrow[from=2-3, to=1-3, "m^*"']
        \end{tikzcd}\]
        We use commutativity of the diagrams on \(\id_Y\) and \(\b{m,f}\) respectively to get \(f;η_Y = α_{X₀}\b{\id_{X₀}, f}\) and \(m;\tmf = α_{X₀}\b{\id_{X₀},f}\).
        It then follows that the inner square commutes.

        Next we wish to show that it is a pullback. Consider \(X \overset{p}{←} P \overset{q}{→} Y\) arbitrary, such that \(p;η_Y = q;\tmf\).

        By similar two naturality diagrams for \(p\) and \(q\) as above we get \(p;η_Y = α_P\p{Fp\p{\id_Y}} = α_P\b{\id_P,p}\) and \(q;\tmf = α_P\b{\bar{m}, \bar{q}f}\), which are equal by assumption.

        (Note: the maps \(\bar{m}\) and \(\bar{q}\) arise from the pullback diagram
        \(\begin{tikzcd}
            {\bar{P}} & {X₀}\\
            {P}       & {X}
            \arrow[from=1-1, to=1-2, "\bar{q}"]
            \arrow[from=1-1, to=2-1, "\bar{m}"]
            \arrow[from=2-1, to=2-2, "q"]
            \arrow[from=1-2, to=2-2, "m"]
            \arrow[from=1-1, to=2-2, "\scalebox{2}{\(\lrcorner\)}"{anchor=center,pos=0},phantom]
        \end{tikzcd}\))

        Since \(α_P\) is an iso it then follows that \(\b{\id_P, p} = \b{\bar{m},\bar{q}f}\).
        Writing out the definition of equivalence we see that \(\bar{m}\) is an iso and \(\bar{m}p = \bar{q}f\). We can then define \(φ ≔ \bar{m}^{-1}\bar{q}\).
        
        The pullback square above gives us the equation \(\bar{q}m = \bar{m}q\).
        Putting it all together we get \(φf = \bar{m}^{-1}\bar{q}f = p\) and \(φm = \bar{m}^{-1}\bar{q}m = \bar{m}^{-1}\bar{m}q = q\), making the square a pullback.

        It remains to show uniqueness of \(\tmf\).
        Let \(t'\) be another map for which
        \(\begin{tikzcd}
            {X₀} & {Y}\\
            {X}  & {\tY}
            \arrow[from=1-1, to=2-1, "m"]
            \arrow[from=1-1, to=1-2, "f"]
            \arrow[from=2-1, to=2-2, "t'"]
            \arrow[from=1-2, to=2-2, "η_Y"]
            \arrow[from=1-1, to=2-2, "\scalebox{2}{\(\lrcorner\)}"{anchor=center,pos=0},phantom]
        \end{tikzcd}\) is a pullback.
        Then, because \(α_X\) is iso, we have \(t' = α_X\b{n,g}\) for some \(\b{n,g} ∈ \Cpar{X,\tY}\).
        Then by the first part of this proof we know that \(\begin{tikzcd}
            {P} & {Y}\\
            {X} & {\tY}
            \arrow[from=1-1, to=2-1, "n"]
            \arrow[from=1-1, to=1-2, "g"]
            \arrow[from=2-1, to=2-2, "t'"]
            \arrow[from=1-2, to=2-2, "η_Y"]
            \arrow[from=1-1, to=2-2, "\scalebox{2}{\(\lrcorner\)}"{anchor=center,pos=0},phantom]
        \end{tikzcd}\) is also a pullback. Because pullbacks are unique up to isomorphism, we have an iso \(X₀ → P\) witnessing \(\p{m,f} ≡ \p{n,g}\), so \(\tmf = α_X\b{m,f} =α_X\b{n,g} = t'\).
    \end{proof}


        % Consider the naturality diagram for \(t'\) applied to \(\b{η_Y,\id_Y}\).
        % % https://q.uiver.app/?q=WzAsOSxbMSwxLCJcXENwYXIoWCxZKSJdLFsyLDIsIlxcQyhaLFopIl0sWzIsMSwiXFxDKFgsWikiXSxbMSwyLCJcXENwYXIoWixZKSJdLFswLDMsIlxcYnvOt19ZLFxcaWRfWX0iXSxbMCwwLCJcXGJ7bSxmfSJdLFsyLDAsInRfe20sZn0iXSxbMywzLCJ0X3vOt19ZLFxcaWRfWX0iXSxbMywxLCJ0Jzt0X3vOt19ZLFxcaWRfWX0iXSxbMCwyXSxbMywxXSxbMywwXSxbMSwyXSxbNCw1LCJGdCciLDIseyJzdHlsZSI6eyJ0YWlsIjp7Im5hbWUiOiJtYXBzIHRvIn19fV0sWzUsNiwizrFfWCIsMix7InN0eWxlIjp7InRhaWwiOnsibmFtZSI6Im1hcHMgdG8ifX19XSxbNyw0LCLOsV9aIiwyLHsic3R5bGUiOnsidGFpbCI6eyJuYW1lIjoibWFwcyB0byJ9fX1dLFs3LDgsInQnXioiLDAseyJzdHlsZSI6eyJ0YWlsIjp7Im5hbWUiOiJtYXBzIHRvIn19fV0sWzYsOCwiIiwyLHsibGV2ZWwiOjIsInN0eWxlIjp7ImhlYWQiOnsibmFtZSI6Im5vbmUifX19XV0=
        % \[\begin{tikzcd}
        %     {\b{m,f}}       &                      & {\tmf} \\
        %                     & {\Cpar{X,Y}}         & {\C{X,\tY}}        & {t';t_{η_Y,\id_Y}} \\
        %                     & {\Cpar{\tY,Y}} & {\C{\tY,\tY}} \\
        %     {\b{η_Y,\id_Y}} &                      &                           & {t_{η_Y,\id_Y}}
        %     \arrow[from=3-2, to=2-2, "{Ft'}"']
        %     \arrow[from=2-2, to=2-3, "{α_X}"']
        %     \arrow[from=3-3, to=2-3, "{t'^*}"]
        %     \arrow[from=3-2, to=3-3, "{α_{\tY}}"']
        %     \arrow[from=4-1, to=1-1, maps to]
        %     \arrow[from=1-1, to=1-3, maps to]
        %     \arrow[from=4-1, to=4-4, maps to]
        %     \arrow[from=4-4, to=2-4, maps to]
        %     \arrow[from=1-3, to=2-4, Rightarrow, no head]
        % \end{tikzcd}\]
        % It remains to show that \(t_{η_Y,\id_Y} = \id_Z\).

        % We are now considering the pullback diagram(s)
        % \[\begin{tikzcd}
        %     {Y}         & {Y}\\
        %     {\tY} & {\tY}
        %     \arrow[from=1-1, to=1-2, "\id_Y"]
        %     \arrow[from=1-1, to=2-1, "η_Y"]
        %     \arrow[from=2-1, to=2-2, "t"']
        %     \arrow[from=1-2, to=2-2, "η_Y"]
        %     \arrow[from=1-1, to=2-2, "\scalebox{2}{\(\lrcorner\)}"{anchor=center,pos=0},phantom]
        % \end{tikzcd}\]
        % with \(t\) standing in for both \(t_{η_Y,\id_Y}\) and \(\id_Z\).
        % Because \(α_Z\) is an isomorphism there exists \(\b{n,g} ∈ \Cpar{\tY,Y}\) such that \(α_Z\b{n,g} = \id_Z\). But then \(X \overset{p}{←} P \overset{q}{→} Y\)
\end{homeworkProblem}
\end{document}
